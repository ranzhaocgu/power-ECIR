
\documentclass[6pt]{article}
\usepackage{float}
\usepackage{graphicx,epstopdf}
\usepackage{bbm}
\usepackage{amsfonts}
\usepackage{mathrsfs}
\usepackage{amsmath}
\usepackage{dsfont}
\usepackage{csquotes}
\usepackage{color}
\usepackage{amssymb}
\usepackage{theorem}
\usepackage{graphicx}
\usepackage[dvips]{epsfig}
\usepackage{latexsym}
\usepackage{exscale}
\usepackage[latin1]{inputenc}
\def\ud{\, \mathrm{d}}
%%%%%%%%%%%%%%%%%%%%%%%%%%%%%%%%%%%%%%%%%%%%%%%%%%%%%%%%%%%%%%%%%%%%%%%%%%%%%%%%%%%%%%%%%%%%%%%%%%%%%%%%%%%%%%%%%%%%%%%%%%%%%%%%%%%%%%%%%%%%%%%%%%%%%%%%%%%%%%%%%%%%%%%%%%%%%%%%%%%%%%%%%%%%%%%%%%%%%%%%%%%%%%%%%%%%%%%%%%%%%%%%%%%%%%%%%%%%%%%%%%%%%%%%%%%%
%TCIDATA{OutputFilter=LATEX.DLL}
%TCIDATA{Version=5.50.0.2890}
%TCIDATA{<META NAME="SaveForMode" CONTENT="1">}
%TCIDATA{BibliographyScheme=Manual}
%TCIDATA{Created=Sunday, September 28, 2014 13:53:13}
%TCIDATA{LastRevised=Monday, October 06, 2014 05:15:04}
%TCIDATA{<META NAME="GraphicsSave" CONTENT="32">}
%TCIDATA{<META NAME="DocumentShell" CONTENT="Standard LaTeX\Blank - Standard LaTeX Article">}
%TCIDATA{CSTFile=40 LaTeX article.cst}

\newtheorem{theorem}{Theorem}[section]
\newtheorem{acknowledgement}[theorem]{Acknowledgement}
\newtheorem{algorithm}[theorem]{Algorithm}
\newtheorem{axiom}[theorem]{Axiom}
\newtheorem{case}[theorem]{Case}
\newtheorem{claim}[theorem]{Claim}
\newtheorem{conclusion}[theorem]{Conclusion}
\newtheorem{condition}[theorem]{Condition}
\newtheorem{conjecture}[theorem]{Conjecture}
\newtheorem{corollary}[theorem]{Corollary}
\newtheorem{criterion}[theorem]{Criterion}
\newtheorem{definition}[theorem]{Definition}
\newtheorem{example}[theorem]{Example}
\newtheorem{exercise}[theorem]{Exercise}
\newtheorem{lemma}[theorem]{Lemma}
\newtheorem{notation}[theorem]{Notation}
\newtheorem{problem}[theorem]{Problem}
\newtheorem{proposition}[theorem]{Proposition}
\newtheorem{remark}[theorem]{Remark}
\newtheorem{solution}[theorem]{Solution}
\newtheorem{summary}[theorem]{Summary}
\newenvironment{proof}[1][Proof]{\noindent\textbf{#1.} }{\ \rule{0.5em}{0.5em}}
\numberwithin{equation}{section}

\begin{document}
\title{On the Distribution of Extended CIR Model}
\author{Qidi Peng\footnote{Institute of Mathematical Sciences, Claremont Graduate University, E-mail: Qidi.Peng@cgu.edu}~~and~ Henry Schellhorn\footnote{Institute of Mathematical Sciences, Claremont Graduate University, E-mail: Henry.Schellhorn@cgu.edu}}
\date{}
\maketitle

\begin{abstract}
We provide a complete representation of the interest rate in the extended
Cox-Ingersoll-Ross\ model. The model we consider is an extension of the
traditonal Cox-Ingersoll-Ross model to the case where all the parameters are
time-varying, and no specific relationship exists between them. The rate can
be represented as the sum of a non-central chi-square random variable, and
of a weighted series of central chi-square random variables, where all
random variables are independent.
\end{abstract}

\section{Introduction}
We consider the extended Cox-Ingersoll-Ross term structure model (ECIR model), namely, the spot interest
rate $r(t)$ is assumed to follow a squared Bessel process $\{r(t)\}_{t\ge0}$ which satisfies the following stochastic differential equation:
\begin{equation}
\label{CIR}
\left\{\begin{array}{ll}
&\ud r(t)=(-b(t)r(t)+\theta(t))\ud t+\sigma(t)\sqrt{r(t)}\ud W(t);\\
&r(0)=r_0\ge0,
\end{array}\right.
\end{equation}
where $b(t)\ge0$, $\sigma(t) >0$ and $\theta(t)\ge0 $ are time dependent continuous functions and $W$ is a standard Wiener process. The Cox-Ingersoll-Ross term structure model (CIR model), was first introduced in Cox et al. (1985a), (1985b).  In its original specification, the speed of mean reversion $b\geq 0$, the volatility $%
\sigma >0$ and the parameter $\theta >0$ are assumed constant.

Several features of the CIR model are particularly attractive. Firstly, it
can be justified by general equilibrium considerations, see Cox et al
(1985a). Secondly, the interest rate is always positive and stationary. Cox
et al. found that its distribution follows a noncentral chi-square
distribution. Finally, there is a closed form formula for the bond price.\
For practitioners however, the main shortcoming of the constant parameters
version of the model is that it cannot reproduce the original term structure
of interest rates. This fact was highlighted by several authors (Hull, 1990; Keller-Ressel and Steiner, 2008; Yang, 2006 and all the references
therein): yield curves can be only normal, inverse, or humped. The extended
CIR model, however, has enough parameters to be fitted to the original yield
curve.

Maghsoodi (1996), Jamshidian (1995), and Rogers (1995) propose a
representation of the extended CIR model as a sum of squares of
Ornstein-Uhlenbeck processes when the dimension $d(t)\equiv 4\theta
(t)/\sigma ^{2}(t)$ is constant and integer. As a consequence, the interest
rate follows the generalized chi-square distribution. Maghsoodi (1996), and
Shirakawa (2002) also propose a representation of the interest rate as a
time-changed lognormal process. However, as the latter author states, \enquote{it is
difficult to derive the probability distribution of the squared Bessel
processes with time-varying dimensions explicitly}. Brigo and Mercurio
(2006) state that no solution to that problem has been found.

In our paper, we determine this distribution explicitly. We show that, for
each, the rate can be represented as the sum of a non-central chi-square
random variable $X_{0}$, and of a weighted series of central chi-square
random variables $\{X_{1},X_{2},\ldots\}$. The variables $X_{0}$, $%
X_{1},X_{2},\ldots$ are all independent. The random variable $X_{0}$ has a
number of degrees of freedom equal to $d(0).$When the dimension $d$ is
constant, the random variables $\{X_{1},X_{2},\ldots\}$ are all zero, and we
recover the traditional result. Thus the latter random variables represent
deviations of the rate from the constant dimension case.

There are alternate approaches to extend the CIR model. Brigo and Mercurio
(2001) find that a deterministic shift of the CIR model is analytically
tractable. The obvious drawback of making the parameters of the CIR model
functions of time is the problem of overparameterization: the parameters
will not be robust in a change of regime. Several authors consider instead a
CIR model of interest rates with constant parameters and stochastic
volatility. For instance, Longstaff and Schwartz (1992), and Duffie and Kan
(1996), consider a generalized two-factor CIR model, where one factor is
the interest rate, and the other one is its volatility. The volatility in
that model is a variation of the volatility in the popular Heston (1993)
model. Cotton et al. (2004) calculate an
asymptotic expansion of the bond price in such a model (with constant
parameters) when the speed of mean reversion is fast. Fouque and Lorig
(2011) generalize this model to a model with a volatility of volatility.

Finally, we note that several authors have generalized the CIR model in a
different way using multiple factors. We refer the reader to the references
contained in Chen, Filipovic and Poor (2004) and Gourieroux and Monfort
(2011).

In the same way that it is not too difficult to generalize the one-factor
CIR model to multiple factors (see Duffie and Kan, 1996), we believe it is
not difficult to generalize our results on the extended CIR model to
multiple factors. However, we leave this for future research.
\section{Characteristic Function of ECIR Model}
\begin{theorem}
\label{thm1}
For $t\ge0$, the characteristic function of $r(t)$ is given by
\begin{equation}  \label{CF}
\mathbb E[e^{i\omega r(t)}]=\exp \bigg( i\omega \Big( \frac{r_{0}e^{-%
\int_{0}^{t}b(u)\ud u}}{1-2i\omega \Sigma (0,t)}+\int_{0}^{t}\frac{\theta
(s)e^{-\int_{s}^{t}b(u)\ud u}}{1-2i\omega \Sigma (s,t)}\ud s\Big) \bigg),
\end{equation}%
where
$$
\Sigma (s,t):=\frac{1}{4}\int_{s}^{t}e^{-\int_{v}^{t}b(u)\ud u}\sigma ^{2}(v)\ud v.
$$
\end{theorem}
\begin{proof}
 Let $\{X^{(\gamma)}_t\}_{t\ge0}$ be a squared Bessel process with initial value $X^{(\gamma)}_0=x$. On one hand, from Proposition 3.4 in Carmona (1996), we derive the characteristic function of $X^{(\gamma)}_t$ as
\begin{equation}
\label{CHX}
\mathbb E[e^{i\omega X_t^{(\gamma)}}]=\exp\bigg(i\omega\Big(\frac{ x}{1-2i\omega t}+\int_0^t\frac{\gamma(u)}{1-2i\omega(t-u)}\ud u\Big)\bigg).
\end{equation}
On the other hand, it is shown that $(r(t))_{t\ge0}$ in (\ref{CF}) follows the same distribution as a squared Bessel process with time and state changes (see Lemma 2.4 and Corollary 3.1 in Shirakawa, 2002):
\begin{equation}
\label{Law}
\left\{r(t)\right\}_{t\ge0}\sim\left\{\nu(t)X_{\tau(t)}^{(\gamma)}\right\}_{t\ge0},
\end{equation}
with
\begin{itemize}
   \item The state change parameter $\nu(t)=\exp\big(-\int_0^tb(u)\ud u\big)$.
  \item The time change parameter $\tau(t)=\frac{1}{4}\int_0^t\frac{\sigma^2(u)}{\nu(u)}\ud u$.
  \item $\gamma(t)=\frac{4(\theta\circ \tau^{-1})(t)}{(\sigma^2\circ \tau^{-1})(t)}$, where $f\circ g$ denotes the composed function of $f$ and $g$.
       \item $\{X_t^{(\gamma)}\}_{t\ge0}$ denotes the squared Bessel process with time-varying dimension $\gamma(t)$ and initial value $X_0^{(\gamma)}=r_0$.
\end{itemize}
It follows from (\ref{CHX}) and (\ref{Law}) that
\begin{eqnarray}
\label{deriveLaw1}
&&\mathbb E[e^{i\omega r(t)}]=\mathbb E\big[e^{i\omega \nu(t)X_{\tau(t)}^{(\gamma)}}\big]\nonumber\\
&&=\exp\bigg(i\omega\nu(t)\Big(\frac{ r_0}{1-2i\omega\nu(t) \tau(t)}+\int_0^{\tau(t)}\frac{4(\theta\circ \tau^{-1})(u)/(\sigma^2\circ \tau^{-1})(u)}{1-2i\omega\nu(t)(\tau(t)-u)}\ud u\Big)\bigg).\nonumber\\
\end{eqnarray}
By the change of variable $u=\tau(s)$ and the fact that $\tau'(s)=\frac{1}{4}\frac{\sigma^2(s)}{\nu(s)}$,
\begin{eqnarray}
\label{deriveLaw2}
&&\int_0^{\tau(t)}\frac{4(\theta\circ \tau^{-1})(u)/(\sigma^2\circ \tau^{-1})(u)}{1-2i\omega\nu(t)(\tau(t)-u)}\ud u
=
\int_0^{t}\frac{4\theta(s)/\sigma^2(s)}{1-2i\omega\nu(t)(\tau(t)-\tau(s))}\tau'(s)\ud s\nonumber\\
&&=\int_0^{t}\frac{\theta(s)/\nu(s)}{1-2i\omega\nu(t)(\tau(t)-\tau(s))}\ud s.
\end{eqnarray}
Then plugging  (\ref{deriveLaw2}) into (\ref{deriveLaw1}) leads to (\ref{CF}).
\end{proof}


Note that another proof of Theorem \ref{thm1} by using the Fokker-Planck equation is provided in Liu, Peng and Schellhorn (2013).
\section{Probability Density of ECIR Model}
In the following theorem we derive the transition probability density of $r(t)$, starting from $r(0)=r_0$. Define
$$
d(t)=\frac{4\theta(t)}{\sigma(t)^2}
$$
and assume $d\in C^1(0,\infty)$.
\begin{theorem}
\label{thm:density}
Let $r(t)$ satisfy Equation (\ref{CIR}), then there is a sequence of independent random variables $\{X_{0},X_{0,N},\ldots,X_{N-1,N},~N\ge1\}$ verifying
\begin{eqnarray*}
&&X_0\sim \chi_{d(0)}^2\Big(\frac{r_0e^{-\int_0^tb(u)\ud u}}{\Sigma(0,t)}\Big);\\
&&X_{j,N}\sim \chi_{\frac{t}{N}d'(\frac{jt}{N})}^2~\mbox{for}~j=0,\ldots,N-1;
\end{eqnarray*}
and
\begin{equation}
\label{dist}
\Sigma(0,t)X_0+\sum_{j=0}^{N-1}\Sigma\Big(\frac{jt}{N},t\Big)X_{j,N}\xrightarrow[N\rightarrow\infty]{\mbox{in distribution}}r(t),
\end{equation}
where we denote by $\chi_q^2(\lambda)$ ($\lambda\ge0$) the noncentral chi-square distribution with $q\ge0$ ($q$ is not necessarily an integer) degrees of freedom and by $\chi^2_q$ with $q\ge0$ a central chi-square distribution with $q\ge0$ degrees of freedom.
\end{theorem}

A straightforward result of Theorem \ref{thm:density} is: the solution of Equation (\ref{CIR}) is distributed as the limit of sum of independent scaled chi-square random variables. More precisely, the following statement holds true:
\begin{corollary}
\label{cor1}
For $t\ge0$, the random variable $r(t)$ in (\ref{CIR}) has probability density (with support $(0,+\infty)$)
\begin{equation}
\label{density}
f_{r(t)}(z)=\lim\limits_{N\rightarrow\infty}h*h_{0,N}*\ldots*h_{N-1,N}(z),
\end{equation}
where $f*g$ denotes the convolution of $f,g$ and the functions $\{h,h_{j,N}:~j=0,1,\ldots,N-1\}$ are defined by: with support $x>0$,
\begin{eqnarray}
\label{h}
&&h(x):=\frac{1}{2\Sigma(0,t)}e^{-\frac{r_0\exp(-\int_0^tb(u)\ud u)+x}{2\Sigma(0,t)}}\Big(\frac{r_0e^{-\int_0^tb(u)\ud u}}{x}\Big)^{1/2-d(0)/4}\nonumber\\
&&\times I_{d(0)/2-1}\Big(\frac{r_0\exp(-\int_0^tb(u)\ud u)x}{\Sigma(0,t)^2}\Big)^{1/2}
\end{eqnarray}
and
\begin{equation}
\label{qJN}
h_{j,N}(x):=\frac{\left(x/(2\Sigma(jt/N,t))\right)^{\frac{t}{2N}d'(\frac{jt}{N})-1}e^{-\frac{x}{2\Sigma(jt/N,t)}}}{2\Sigma(jt/N,t)\Gamma\left(\frac{t}{2N}d'(\frac{jt}{N})\right)},
\end{equation}
where
\begin{itemize}
\item $I_{\alpha}$, with $\alpha$ being integer or positive, is the  modified Bessel function of the first kind defined by (see e.g. Abramowitz and Stegun, 1965)
$$
I_\alpha(x):=\sum_{l=0}^{\infty}\frac{(\frac{x}{2})^{2l+\alpha}}{l!\Gamma(l+\alpha+1)}.
$$
\end{itemize}
\end{corollary}
Now we prove Theorem \ref{thm:density}.


\begin{proof}
First we observe that, by using integration by parts,
\begin{eqnarray}
\label{intd}
&&\frac{1}{2}\int_0^td'(s)\log\left(1-2i\omega\Sigma(s,t)\right)\ud s\nonumber\\
&&=-\frac{d(0)}{2}\log\left(1-2i\omega\Sigma(0,t)\right)-i\omega\int_0^t\frac{d(s)e^{-\int_s^tb(u)\ud u}\sigma^2(s)/4}{1-2i\omega\Sigma(s,t)}\ud s.
\end{eqnarray}
Since the mappings $s\mapsto d'(s)$ and $s\mapsto\Sigma(s,t)$ are continuous, we use (\ref{CF}), (\ref{intd}) and the Riemann sum representation of integral to obtain
\begin{eqnarray}
\label{E:dec}
\mathbb E[e^{i\omega r(t)}]&=&\frac{\exp\big(\frac{i\omega r_0e^{-\int_0^tb(u)\ud u}}{1-2i\omega \Sigma(0,t)}-\frac{1}{2}\int_0^td'(s)\log\left(1-2i\omega\Sigma(s,t)\right)\ud s\big)}{(1-2i\omega \Sigma(0,t))^{d(0)/2}}\nonumber\\
&=&\frac{\exp\big(\frac{i\omega r_0e^{-\int_0^tb(u)\ud u}}{1-2i\omega \Sigma(0,t)}\big)}{(1-2i\omega \Sigma(0,t))^{d(0)/2}}\cdot e^{-\frac{1}{2}\int_0^td'(s)\log\left(1-2i\omega\Sigma(s,t)\right)\ud s}\nonumber\\
&=&\frac{\exp\big(\frac{i\omega r_0e^{-\int_0^tb(u)\ud u}}{1-2i\omega \Sigma(0,t)}\big)}{(1-2i\omega \Sigma(0,t))^{d(0)/2}}\Big(\lim_{N\rightarrow\infty}e^{-\frac{1}{2}\sum_{j=0}^{N-1}\frac{t}{N}d'(\frac{jt}{N})\log(1-2i\omega\Sigma(\frac{jt}{N},t))}\Big)\nonumber\\
&=&\lim_{N\rightarrow\infty}\frac{\exp\big(\frac{i\omega r_0e^{-\int_0^tb(u)\ud u}}{1-2i\omega \Sigma(0,t)}\big)}{(1-2i\omega \Sigma(0,t))^{d(0)/2}}\Big(\prod_{j=0}^{N-1}\Big(1-2i\omega\Sigma(\frac{jt}{N},t)\Big)^{-\frac{t}{2N}d'(\frac{jt}{N})}\Big).\nonumber\\
\end{eqnarray}
The first factor on the right-hand side of (\ref{E:dec}):
$$
\frac{\exp\big(\frac{i\omega r_0e^{-\int_0^tb(u)\ud u}}{1-2i\omega \Sigma(0,t)}\big)}{(1-2i\omega \Sigma(0,t))^{d(0)/2}}
$$
is known as the characteristic function of a scaled noncentral chi-square distribution with noncentral parameter $\frac{r_0e^{-\int_0^tb(u)\ud u}}{\Sigma(0,t)}$, $d(0)$ degrees of freedom and scaling parameter $\Sigma(0,t)$, i.e., its corresponding random variable is $\Sigma(0,t)X_0$ with
\begin{equation}
\label{X0}
X_0\sim\chi_{d(0)}^2\Big(\frac{r_0e^{-\int_0^tb(u)\ud u}}{\Sigma(0,t)}\Big).
\end{equation}
For the second part on the right-hand side of (\ref{E:dec}), we observe that for $j=0,1,\ldots,N-1$, $\omega\mapsto(1-2i\omega\Sigma(\frac{jt}{N},t))^{-\frac{t}{2N}d'(\frac{jt}{N})}$ is the characteristic function of the random variable $\Sigma(\frac{jt}{N},t)X_{j,N}$, with
\begin{equation}
\label{XjN}
X_{j,N}\sim \chi^2_{\frac{t}{N}d'(\frac{jt}{N})}.
\end{equation} Therefore Theorem \ref{thm:density} follows from (\ref{E:dec}), (\ref{X0}), (\ref{XjN}) and the L\'evy-Cram\'er continuity theorem.
\end{proof}
\section{An approximation formula of the density function of $r(t)$}
In this section, we derive an efficient approximation formula of the explicit probability density of $r(t)$. The result is the following theorem.
 \begin{theorem}
 \label{exdensity}
 Let $N\ge2$ and
 \begin{equation}
 \label{exfN}
 f_N=h*h_{0,N}*\ldots*h_{N-1,N}.
 \end{equation}
We have for $z>0$,
 \begin{eqnarray}
 \label{exfN1}
&&f_N(z)=\int_{0}^z\frac{1}{2\Sigma(0,t)}e^{-\frac{r_0\exp(-\int_0^tb(u)\ud u)+x}{2\Sigma(0,t)}}\Big(\frac{r_0e^{-\int_0^tb(u)\ud u}}{x}\Big)^{1/2-d(0)/4}\nonumber\\
&&\times I_{d(0)/2-1}\Big(\frac{(r_0x)^{1/2}\exp(-\frac{1}{2}\int_0^tb(u)\ud u)}{\Sigma(0,t)}\Big)\Big(\sum_{k=0}^{\infty}b_{k,N}\psi_{k,N}(z-x)\Big)\ud x,\nonumber\\
\end{eqnarray}
where
\begin{itemize}
\item for $N\ge2$ and $k\in\mathbb N$,
$$
b_{k,N}=a_{k,N}\prod_{j=1}^{N-1}\big(\frac{\Sigma(0,t)}{\Sigma(\frac{jt}{N},t)}\big)^{\frac{t}{2N}d'(\frac{jt}{N})}
$$
and the sequence $\{a_{k,N}\}_{k\ge0}$ satisfies
$$
a_{k,2}=\Big(\frac{t}{4}d'(\frac{t}{2})\Big)^{(k)}\Big(1-\frac{\Sigma(0,t)}{\Sigma(\frac{t}{2},t)}\Big)^kk!,~\mbox{for $k\in\mathbb N$}
$$
and
$$
a_{k,N}=\sum\limits_{l=0}^kA_{l,N}^{(N-1)}\Big(\frac{t}{2N}d'(\frac{(N-1)t}{N})\Big)^{(k-l)}\Big(1-\frac{\Sigma(0,t)}{\Sigma(\frac{(N-1)t}{N},t)}\Big)^{k-l}(k-l)!
$$
for $N\ge3$ and $k\in\mathbb N$, where $(m)^{(r)}=m(m+1)\ldots(m+r-1)$ denotes the rising factorial.
\item $\psi_{k,N}$ is the density function of the Gamma distribution with shape parameter $\sum_{j=0}^{N-1}\frac{t}{2N}d'(\frac{jt}{N})+k$ and scale parameter $2\Sigma(0,t)$:
\begin{equation}
\label{psi}
\psi_{k,N}(x)=\frac{x^{\sum_{j=0}^{N-1}\frac{t}{2N}d'(\frac{jt}{N})+k-1}e^{-\frac{x}{2\Sigma(0,t)}}}{(2\Sigma(0,t))^{\sum_{j=0}^{N-1}\frac{t}{2N}d'(\frac{jt}{N})+k}\Gamma\big(\sum_{j=0}^{N-1}\frac{t}{2N}d'(\frac{jt}{N})+k\big)}.
\end{equation}
\end{itemize}
 \end{theorem}
 \begin{proof} In view of Moschopoulos and Canada (1984), the probability density $f_N$ in (\ref{exfN}) is given as
\begin{equation}
\label{df}
f_N(z)=\Big(\prod_{j=1}^{N-1}\Big(\frac{\Sigma(0,t)}{\Sigma(\frac{jt}{N},t)}\Big)^{\frac{t}{2N}d'(\frac{jt}{N})}\Big)\sum_{k=0}^{\infty}a_{k,N}\psi_{k,N}(z),
\end{equation}
where
\begin{itemize}
\item the sequence $\{a_{k,N}\}_{k\ge0,N\ge2}$ solves
$$
\prod_{k=2}^{N}\Big(\sum_{r=0}^{\infty}(\frac{t}{2N}d'(\frac{(k-1)t}{N}))^{(r)}\big(1-\frac{\Sigma(0,t)}{\Sigma(\frac{(k-1)t}{N},t)}\big)^{r}r!x^{-r}\Big)=\sum_{k=0}^{\infty}a_{k,N}x^{-k},
$$
Equivalently (we refer to (2.3), (2.5) in Moschopoulos and Canada, 1984), for $N\ge 2$,
$a_{k,N}=A_{k,N}^{(N)}$, with $\{A_{k,N}^{(p)}\}_{k\ge0,p\ge2}$ being defined by
$$
A_{k,N}^{(2)}=\Big(\frac{t}{2N}d'(\frac{t}{N})\Big)^{(k)}\Big(1-\frac{\Sigma(0,t)}{\Sigma(\frac{t}{N},t)}\Big)^kk!,~\mbox{for $k\in\mathbb N$}
$$
and
$$
A_{k,N}^{(p)}=\sum\limits_{l=0}^kA_{l,N}^{(p-1)}\Big(\frac{t}{2N}d'(\frac{(p-1)t}{N})\Big)^{(k-l)}\Big(1-\frac{\Sigma(0,t)}{\Sigma(\frac{(p-1)t}{N},t)}\Big)^{k-l}(k-l)!
$$
for $p\ge3$ and $k\in\mathbb N$.
\item The expression of $\psi_{k,N}$ in (\ref{psi}) results from (2.4) in Moschopoulos and Canada (1984).
\end{itemize}
Finally (\ref{exfN1}) follows from (\ref{h}) and (2.6) in Moschopoulos and Canada (1984).
\end{proof}
\begin{thebibliography}{99}
\footnotesize

\bibitem{Abramowitz and Stegun}
{\sc Abramowitz, M. and Stegun, I.} (1965). {\em Handbook of Mathematical Functions: with Formulas, Graphs, and Mathematical Tables.} New York, Dover 13.

\bibitem{Bell}
{\sc Bell, E.T.} (1934). Exponential polynomials. {\em Annals of Mathematics.} {\bf 35,} 258--277.
\bibitem{Carmona} {\sc Carmona, P.} (1994) G\'en\'eralisations de la loi de l'arc sinus et entrelacements de processus de
Markov, {\em Ph.D. dissertation, University Paris 6.}

\bibitem{Cochran}
{\sc Cochran, W.~G.} (1934). The distribution of quadratic forms in a normal
system, with applications to the analysis of covariance. {\em Mathematical
Proceedings of the Cambridge Philosophical Society.} {\bf 30(2),} 178--191.

\bibitem{Comtet}
{\sc Comptet, L.} (1974). \emph{Advanced Combinatorics: The Art of Finite and Infinite Expansions,} Revised and Enlarged Edition, D. Reidel Publishing Co., Dordrecht and Boston.

\bibitem{Cox et al.}
{\sc Cox, J.~C., Ingersoll, J.~E. and Ross, S.~A.} (1985).  A theory of the term
structure of interest rates. {\em Econometrica.}
{\bf 53,} 385--487.

\bibitem{Daalhuis}
{\sc Daalhuis, A.~B.} (2010).  {\em Confluent hypergeometric function, in: Handbook of Mathematical Functions.} Cambridge University Press.

\bibitem{Daboul}
{\sc Daboul, S. et al.} (2013).  The Lah numbers and the $n$th derivative of $Exp(1/x)$.   \emph{Mathematics Magazine,} \textbf{86 (1),} 39--47.


\bibitem{Evans et al.}
{\sc Evans, M., Hastings, N. and Peacock, B.} (1992). {\em Statistical Distributions,} 3rd ed, New York, Wiley, pp. 57.

\bibitem{Liu et al.}
{\sc Liu, Z., Peng, Q. and Schellhorn, H.} (2013).  A bond option pricing formula
in a generalized CIR model with stochastic volatility. Preprint.


\bibitem{Maghsoodi}
{\sc Maghsoodi, Y.} (2006).  Solutions of the extended CIR term structure and
bond option valuation. {\em Mathematical Finance}
{\bf 6(1),} 89--109.

\bibitem{Moschopoulos}
{\sc Moschopoulos, P.G. and Canada, W.B.} (1984).  The distribution function of a linear combination of chi-squares. {\em Comp. $\&$ Maths. with Appls.}
{\bf 10(415),} 383--386.

\bibitem{Siegel}
{\sc Siegel, A. F.} (1979).  The noncentral chi-square distribution with zero degrees of freedom and testing for uniformity. {\em Biometrika}
{\bf 66(2),} 381--386.

\bibitem{Riordan}
{\sc Riordan, J.} (1958).  \emph{Introduction to Combinatorial Analysis,} Princeton University Press.

\bibitem{Shirakawa}
{\sc Shirakawa, H.} (2002).  Squared Bessel processes and their applications
to the square root interest rate model {\em Asia-Pacific Financial Markets.}
{\bf 9,} 169--190.

\bibitem{Spanier}
{\sc Spanier, J. and Oldham, K. B.} (1987).  \emph{An Atlas of Functions,} Hemisphere Publishing Co., Sringer-Verlag.

\end{thebibliography}
\end{document}
